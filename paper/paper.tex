% !TeX root = ./paper.tex
\documentclass[modern]{aastex62}

% Load the corTeX style definitions
\input{cortex}

% Bibliography stuff
\bibliographystyle{aasjournal}

% Begin!
\begin{document}

% Title
%\title{Inferring a time dependent map of Io's surface from occultations and phase curves.}
\title{Time-variable surface mapping of exoplanets: an application to Io}

% Author list
\author{Fran Bartoli\'c}
\email{fbartolic@flatironinstitute.org}
\affil{Center~for~Computational~Astrophysics, Flatiron~Institute, New~York, NY}
\affil{Centre for Exoplanet Science, SUPA, School of Physics and Astronomy, University of St. Andrews, St. Andrews, UK}
\author{Rodrigo Luger}
\author{Daniel Foreman-Mackey}
\affil{Center~for~Computational~Astrophysics, Flatiron~Institute, New~York, NY}
%

\begin{abstract}
Jupiter's moon Io is the most volcanically active body in the Solar System with hundreds of active volcanoes varying in intensity on different timescales.
Existing photometric observations in the near infrared taken during occultations by Jupiter and Galilean moons encode a wealth of information about its surface.
As such, Io is an ideal testbed for models of time-variable exoplanet surfaces.
We build a single probabilistic model of Io's surface using Nonnegative Matrix Factorization to infer one map per light curve.
The model is built on top of the code starry \href{https://rodluger.github.io/starry/}{\color{linkcolor}\faGithub} which can compute occultation light curves and phase curves in both emitted and reflected light. It also enables efficient inference with gradient based samplers.
We use the inferred map to study the time variability of prominent volcanoes over a timescale of decades.
While we mostly focus on Io, the lessons we learn are directly applicable to mapping exoplanets.
We discuss this application towards the end of the paper.\href{https://github.com/fbartolic/volcano}{\color{linkcolor}\faGithub}

\end{abstract}

%
\section{Introduction}
%% ADD PARAGRAPH REVIEWING PREVIOUS LITERATURE ON MAPPING EXOPLANETS, MENTION JWST AND LUVOIR
%% MAYBE EVEN IN THE ABSTRACT
The surface of Jupiter's moon Io is littered with hundreds of volcanoes appearing as bright spots in the near infrared and varying in intensity on timescales of days to decades.
The volcanic activity is driven by tidal interactions with Jupiter and sustained by the Laplace resonance with Europa and Ganymede \citep{peale_melting_1979}.
Io is a perfect analogue of an extremely volcanically active exoplanet.
Such rocky exoplanets, often dubbed \emph{super-Ios} or \emph{lava worlds}, have gathered a lot of interest in recent years because the photometric and spectral signatures of volcanic activity on such worlds will likely be detectable in the near future \citep{kaltenegger_detecting_2010,henning_highly_2018,oza_sodium_2019}.
Studying volcanism on both Io and exoplanets is scientifically valuable because it provides a window into the properties of early Earth and it informs theories of planet formation.

Existing exoplanet detections with potential volcanism include CoRoT-7b \citep{barnes_corot-7b_2010}, the first discovered rocky exoplanet with likely significant tidal heating; 55 Cancri e, whose longitudinal offset in peak surface emission has been attributed to, among other things, lava flows on the surface \citep{demory_variability_2016,demory_map_2016,hammond_linking_2017}; several planets in the TRAPPIST-1 system in a Laplace-like resonance with likely volcanic activity \citep{kislyakova_magma_2017,dobos_tidal_2019}; and many others.
Not all exoplanets will have Io-like volcanism, some will have magma oceans due to their proximity to the star or volcanism caused by nuclear decay similar to early Earth volcanism.

Io has been observed extensively using both space and ground observatories.
High resolution images of Io's surface were taken by space missions such as Voyager \citep{smith_jupiter_1979}, Galileo \citep{belton_galileos_1996} and Juno \citep{mura_infrared_2020}.
It has also been resolved from the ground in the near infrared using disk resolved imaging
\citep{howell_infrared_1985,simonelli_disk-resolved_1986,spencer_discovery_1990} and adaptive optics observations \citep{marchis_adaptive_2000,marchis_keck_2005,de_kleer_spatial_2016}.
Most importantly for this work, Io has been sporadically observed for decades using high cadence near infrared photometry taken during occultations by Jupiter starting with \cite{spencer_discovery_1990}.
Occultations by Jupiter occur twice every orbit lasting ~1.7 days and mutual occultations with Europa, Ganymede or Callisto happen every ~6 years when Earth pasess through the orbital plane of Galilean satellites.
The majority of occultations by Jupiter are observed when Io is in Jupiter's shadow ("in eclipse") while mutual occultations are almost always observed in sunlight because mutual occultations while Io is in eclipse are extraordinarily rare.
Only the brightest volcanoes are visible over the reflected light background when Io is illuminated by the Sun\citep{veeder_ios_1994,de_kleer_spatial_2016}.

Both kinds of occultations have been used to study volcanic activity on the surface.
\cite{spencer_io_1994} observed several occultations of Io by Europa and detected a major brightening of the most powerful of Io's volcanoes, Loki, relative to previous Voyager observations.
\cite{rathbun_loki_2002, rathbun_loki_2006,rathbun_ground-based_2010} used observations from NASA's IRTF telescope to study the long term variability of different volcanic spots finding evidence of periodicity in Loki's eruptions and established the transient nature of the volcanoes (only Loki is persistently active).
By studying an occultation of Io by Europa, \cite{de_kleer_multi-phase_2017} mapped the Loki region to a precision of about 2 kilometers.
In addition to observations of occultations in near infrared, several groups have been observing mutual occultations in the optical \citep[][and references therein]{arlot_mutual_1974,saquet_phemu15_2018,morgado_astrometry_2016} with the purpose of inferring the optical albedo of Io and improving ephemeris precision for Galilean satellites.

None of the studies of occultations in the infrared attempted to infer a two-dimensional map of Io's surface.
The light curves are usually fit independently with the goal of inferring one-dimensional longitudinal variations in brightness on its surface.
Strong priors on the locations of individual volcanoes are assumed based on maps from space mission observations.
In this work, we build a fully probabilistic model in order to infer a time-dependent two-dimensional \emph{map} of Io's volcanic surface from archival IRTF observations in the near infra red.
The model is built on top of the recently developed code starry \citep[][Luger et al. 2020 in prep]{luger_starry_2019} which
enables fast analytic computation of occultation light curves and phase curves for objects whose surface features are represented in terms of spherical harmonics.
starry can compute phase curves and occultations in both emitted light (for modeling the isotropic volcanic emission from Io's surface) and reflected light (for mapping albedo variations).
Instead of fitting one map per occultation light curve, we reduce the dimensionality of the problem using matrix factorization, assuming a fixed map for a given light curve which is a linear combination of $K$ basis maps.
We use priors on the map features which ensure the maps are physical and prevent overfitting of the data.
The model is interpretable, each of the basis maps roughly corresponds to a single spot on the surface, and the coefficients multiplying the basis maps encode information on the time variability.

Our goals for this paper are to both improve on past studies of Io using occultation light curves and to construct a flexible model directly applicable to future efforts on inferring time-dependent maps of variable features on exoplanet surfaces such as volcanoes, magma oceans or clouds.
The fact that we have lots of prior information on what the surface of Io should look like enables us to test our model in a realistic setting, testing sensitivity to chosen priors, data quality, the orbital parameters and different kinds of surface features.

The paper is organized as follows.
In \S\ref{sec:data} we describe the IRTF light curves of occultations by Jupiter which we use to infer maps.
In \S\ref{sec:model} we discuss the generative model for the data for the case of a static map, and a dynamic map modeled using a matrix factorization approach and we write down the relevant likelihood functions.
In \S\ref{sec:inverse_problem} we focus on the question of how to do Bayesian inference given the forward model specifications defined in \S\ref{sec:model}.
We discuss and test different priors on map features, the information content of light curves and fitting the models using variational inference (VI) methods and Hamiltonian Monte Carlo (HMC).
We present results on simulated data.
\S\ref{sec:results} shows the results on real data.
In particular, the inferred dynamic map, the  different component maps and accompanying coefficients, the uncertainties of the maps, sensitivity to different choices of priors, comparison of inferred locations and intensities of spots to known values from the literature.
Finally, in \S\ref{sec:exoplanets} we discuss applications of the model to future exoplanet obervations and test the model predictions on simulated JWST observations of a volcanic Super-Earth.

\section{Data}
\label{sec:data}

\begin{figure}[h!]
    \begin{centering}
    \includegraphics[width=0.5\linewidth]{figures/irtf_max_flux.pdf}
    \oscaption{irtf_dataset}{%
        Maximum flux for each curve taken using NASA's IRTF telescope during an
        occultation of Io by Jupiter.
       \label{fig:irtf_max_flux}
    }
    \end{centering}
\end{figure}

The dataset we use to infer maps of Io's surface consists of 112 observations of Io in the near infrared taken from 1996 until 2018 using various instruments at NASA's IRTF observatory.
Figure~\ref{fig:irtf_max_flux} shows the maximum flux in $\mathrm{GW}/\mu \mathrm{m}/\mathrm{sr}$ for each light curve plotted as a function of time which approximately corresponds to the disc integrated flux of Io at the beginning of ingress or the end of egress prior to it being occulted by Jupiter.
Two features of this plot are apparent.
First although the observations span decades the cadence is non-uniform with only a few observations taken between 2008 and 2016.
Second, the baseline brightness is varying stochastically by a large amount and there are several notable events of increased volcanic activity.
All of the observations were taken while Io was \emph{in eclipse}, meaning that it was in Jupiter's shadow and all of the observed emission from the surface is due to thermal radiation.
Observations of Io \emph{in sunlight} on the other hand probe both the thermal (volcanic) emission and the surface albedo variations in the near infrared.


\begin{figure}[h!]
    \begin{centering}
    \includegraphics[width=0.5\linewidth]{figures/irtf_sample_lightcurves.pdf}
    \oscaption{irtf_dataset}{%
        A selection of sample light curves taken during occultations of Io by Jupiter in our dataset.
        The step-like morphology of the light curves is due to bright volcanoes on Io's surface coming in or out of view during an occultation.
        These light curves thus visibly encode information about the features on the surface.
        \label{fig:irtf_sample_lightcurves}
    }
    \end{centering}
\end{figure}

In Figure~\ref{fig:irtf_sample_lightcurves} we display a random subset of all the light curves
in the dataset.
All occultations last for $\sim4$ min and the cadence for each light varies but it on the order of a second.
It follows that over the course of a single exposure Jupiter's limb move by about 15km which provides a lower bound for the size of the features on the surface that we can reliably estimate.
The shapes of light curves strongly deviate from the smooth variability one would expect assuming a homogeneous distribution of thermal emission on the surface.
Especially prominent are light curves with clear step-like features which are present when bright spots come in or out of view during the course of an occultation.
The fact that these features are so clearly visible means that even individual light curves encode a wealth of information about the surface.

The photometric quality varies from year to year because multiple instruments were used over the years.
An additional issue is that estimated errorbars aren't provided.
As with all ground based photometry, the observations are influenced by atmospheric variability which induces correlated noise in the light curves.
Because of this the flux isn't always monotonically increasing or decreasing as one would expect.
All of these issues need to be accounted for in the final model.

\section{Model}
\label{sec:model}
\subsection{Orbital parameters}
\label{ssec:orbital_parameters}
To model the occultations we need to know the projected separation between Jupiter and Io and the orientation of Io at any given time.
We use the \href{https://ssd.jpl.nasa.gov/horizons.cgi}{JPL Horizons database} and the latest planetary ephemeris DE430 \citep{folkner_planetary_2014} which are accurate to a few kilometers on Io's surface.
Specifically, we use the Right Ascension and Declination, the angular (equatorial) diameter (\textsf{Ang-diam}), the visibility flag indicating whether Io is occulted or not and whether it is in partial or total eclipse (\textsf{Ang-sep/v}), the longitude and latitude at the center of Io's disc as seen from Earth (\textsf{Ob-lon}, \textsf{Ob-lat}), the sub-solar point longitude and latitude which specify the direction pointing towards the Sun on Io's surface (\textsf{Sl-lon}, \textsf{Sl-lat}), the distance from Io to the Sun (\textsf{r}) and
the counterclockwise angle between the Celestial North Pole unit vector projected onto the plane of the sky and the Io's north pole (\textsf{NP.ang}).
All longitudes are positive to the west.
\textsf{HORIZONS} provides all ephemeris with a minimum cadence of 1 second so we interpolate all values such that we can evaluate them for arbitrary times.

The coordinate system in \textsf{starry} is defined to be right handed such that the $\hat{z}$
axis points towards the observer and the $\hat{x}$ axis points to the right on the plane of the sky.
The radius of the occulted sphere is fixed to 1 and the orientation is specified by three angles,
the counterclockwise obliquity angle \textsf{obl} between the $\hat{y}$ axis and the North Pole of the sphere, the inclination angle \textsf{inc} which is set to $90^\circ$ if the North Pole is aligned with the $\hat{y}$ axis and the phase angle \textsf{theta} rotates the sphere around the $\hat{y}$ axis in the eastward direction.
We have $\mathrm{obl}=\textsf{NP.ang}$, $\mathrm{inc}=90^\circ-\textsf{Ob-lat}$ and
$\mathrm{theta}=\textsf{Ob-lon}$.
The occultor position relative to the occulted object is given by
\begin{align}
    \mathrm{xo}/\gamma&=-\Delta\alpha\,\cos\delta\\
    \mathrm{yo}/\gamma&=\Delta\delta\\
    \mathrm{zo}/\gamma&=1
\end{align}
where $\Delta\alpha$ are the $\Delta\delta$ differences in Right Ascension and Declination respectively relative to the occulted object and $\gamma$ is the angular radius of the occulted sphere.
If the occulted object is illuminated by the Sun then we also need to specify the coordinates of the light source which are given by
\begin{align}
    \mathrm{xs}&=r\cos\theta\sin\phi\\
    \mathrm{ys}&=r\sin\theta\\
    \mathrm{zs}&=r\cos\theta\cos\phi
\end{align}
where $\theta$ is obtained by subtracting \textsf{Ob-lat} from \textsf{Sl-lat} and similarly $\phi$ is obtained by subtracting \textsf{Ob-lon} from \textsf{Sl-lon}.
$r$ is equal to the heliocentric distance of the occulted object \textsf{r}.

\subsection{Effective radius of Jupiter}
Although the sky position of Jupiter relative to Io is known to a precision of a few kilometers, several complications arise when attempting to compute Jupiter's radius.
First, Jupiter is not spherical and its equatorial radius is larger than the polar radius by thousands of kilometers. 
Because \textsf{starry} does not support occultations by non-spherical occultors we 
assume that Jupiter is locally spherical at the point of an occultation and estimate an effective radius from measurements of Jupiter's shape which are based on Voyager radio occultation measurements \citep{lindal1981}.
These data come in the form of a plot of effective radius  as a function of planetocentric latitude at a fixed pressure of 100 mbar \citep[Fig.~7 in ][]{lindal1981}.
When computing the occultation latitude we ignore the variation in effective radius along Jupiter's limb projected on Io's surface and instead fix Jupiter's latitude to the value corresponding to the center of Io's disc.

The second complication is that Jupiter is gaseous which means that it does not have a well defined boundary.
In principle, we should compute an effective radius of Jupiter at different locations (pressure) in the atmosphere and model an occultation of Io by a fuzzy occultor.
Although this can be done with Starry, it is not necessary because the characteristic scale height of Jupiter is around 27 km (CITE) which is below the uncertainty of our inferred maps.
Instead, we follow \cite{spencer1990} and compute the effective radius of Jupiter at about 2.2 mbar, this is the pressure (and the associated effective radius) at which during an occultation a bright source on the surface of Io fades by 50\% due to differential refraction.

Since the effective radius information is provided only at a fixed pressure of 100 mbar, 
to adjust those values for a lower pressure of 2.2 mbar, we assume an exponential pressure profile $P=P_0\, e^{-\Delta r/H}$ where $H$ is the scale height and $\Delta r$ is the height difference between the two pressure levels.
It follows that to convert the shape profile at 100 mbar to 2.2 mbar we need to add the factor
$-H\ln(2.2/100)$ which is assumed to be constant in the $\pm 21$ deg latitude range in which the occultations occur.
Besides refractive absorption there is a also a slight additional methane molecular absorption which \cite{spencer1990} estimate to be equal to around 12\% in their filter, 
we choose to ignore this because it is far below the resolution of our maps.

Finally, we have to account for the fact that the light from Io is getting significantly bent at the point of half reflective intensity in Jupiter's atmosphere 
which results in a smaller shadow of Jupiter on the surface of Io then would be the case if the light didn't get bent.
It is zero at the beginning of the disappearance of a hot spot when there is no refraction, increasing to one scale height $H$ at the half intensity point and then increasing further to a large value when Io disappears behind Jupiter's limb.
We ignore the variation in the bending and adopt a fixed value of one scale height for this effect which we subtract from the value of the effective radius.

In summary, to compute an effective radius of Jupiter for a given occultation of Io we first have to compute the latitude at which the occultation is occurring, use a modified shape profile from \cite{lindal1981} to get an effective radius and finally subtract one scale height due to light bending.
There are substantial uncertainties in each of these steps, most notably the shape profile data are quite old and it is not clear that the structure of Jupiter's atmosphere stayed the same since 1980s. 
The shape profile also depends on the wind velocity structure and temperature which we ignore.
In addition to uncertainties about the atmospheric structure, there are uncertainties associated with digitizing the data shown in Fig.~7 in \cite{lindal1981} because it is not available in table form.
We digitizing this figure by reading off the data points "by eye" and then interpolating them using cubic spline interpolation.
Finally we add the pressure conversion factor and subtract one scale height due to light bending.

Throughout this paper the highest resolution maps we fit to the data are of order $l=25$ which corresponds to a characteristic length scale of around 230 km on the surface of Io.
Quantifying the uncertainty in each of these steps is beyond the scope of this paper but we are confident that the errors in the radius estimate are below the typical resolution of our inferred maps. 
We provide justification for this assumption in section~\ref{ssec:static_map} where we infer the location of Loki and it ends up matching well with the high resolution geological map of Io.

\subsection{Static map model}
\label{ssec:static}
Given the geometry of on occultation event at an arbitrary time, computing the predicted flux with \textsf{starry} is straightforward.
\textsf{starry} computes the integrated flux of an unocculted or an occulted sphere analytically assuming that the surface map can be expanded in terms of spherical harmonics $Y_{lm}(\theta,\phi)$ up to a certain \emph{degree} $l$.
The map is thus defined by a vector of spherical harmonic coefficients $\mathbf{y}$ multiplying the spherical harmonic basis $\tilde{\mathbf{y}}(x, y)=\left(Y_{0,0} Y_{1,-1} Y_{1,0} Y_{1,1} Y_{2,-2} Y_{2,-1} Y_{2,0} Y_{2,1} Y_{2,2} \cdots\right)^{\top}$ and the total number of coefficients is $(l+1)^2$.
In addition to computing \emph{thermal} phase curves and occultation light curves, \textsf{starry} also solves the considerably more complex problem of computing reflected light phase curves and occultations when the sphere is illuminated by a distant light source \citep[Luger et al. 2020 in prep][]{} which is complicated by the presence of the terminator line.
In the case of a reflected light map, the coefficient vector $\mathbf{y}$ represents spherical albedo in a given wavelength range.
Most importantly, conditional on the parameters specifying the geometry of the occultations being fixed, the \textsf{starry} model is \emph{linear} for both emitted and reflected light maps.
This is achieved by representing all rotations, changes of basis and integrals with complicated boundaries needed to compute the flux as linear transformations.
Since a sequence of linear mappings is also linear, the predicted flux can be written as
\begin{equation}
    \mathbf{f}=\mathbf{A}\,\mathbf{y}
    \label{eq:linear_model}
\end{equation}
where the column vector $\mathbf{f}$ of shape $(T, 1)$ is the predicted flux for different values of the occultor position and optionally the direction of the illumination source and $\mathbf{A}$ is the design
matrix of shape $(T, N)$ (with $N=(l+1)^2$) which encodes information all operations needed to compute the integrated flux.
If the geometry isn't known precisely then the matrix $\mathbf{A}$ is not fixed and the model is no longer linear.

The characteristic angular scale of features that can be represented with a given map is set by the degree of the map and it is approximately equal to $\frac{180^\circ}{l}$.
State of the art inferences involving phase curves and secondary eclipses of exoplanets are able to constrain features of order $l=1$ (inferring a bright spot on single hemisphere) but for Io we need to fit much higher order maps because the typical scale of volcanic spots is on the order of tens of kilometers (a few degrees).
\textsf{starry} can handle computations with maps up to $l\approx 20$ before numerical instabilities kick in (TODO: make this statement more precise).

Although \textsf{starry} is built around the idea of expanding surface features in a spherical harmonic basis, this is not the ideal basis for doing inference because it makes it difficult to ensure that the intensity is positive everywhere on the map.
In order to circumvent this limitation of the spherical harmonic basis, we fit all models in a \emph{pixel basis} instead where ensuring positivity amounts to putting a positive prior on the pixels but still do all computations in the $Y_{lm}$ basis where all integrals are analytic and can be computed efficiently.
We obtain the pixels by constructing a discrete grid on the surface of the sphere using an equal area Molleweide projection
and to ensure that the intensity is positive everywhere on the surface we need to have more pixels than spherical harmonics (by a factor of a few).
Given a vector of spherical harmonic coefficients we can switch to the pixel basis with a linear operator $\mathbf{P}$ which is straightforward to compute.
The inverse transform is less straightforward because the number of pixels is greater than the number of $Y_{lm}$ coefficients so the matrix $\mathbf{P}$ is not square and is therefore not invertible.
Although we can't invert this matrix we can still compute the Moore-Penrose pseudoinverse $\mathbf{P}^\dagger$ which is obtained by solving the linear system $\mathbf{P}\,\mathbf{y}=\mathbf{p}$ where $\mathbf{p}$ is the vector of all pixels comprising a map. 
The flux can then be computed as a linear operation on the pixels: 
\begin{equation}
    \mathbf{f}=\mathbf{A}\,\mathbf{P}^\dagger\,\mathbf{p}'
    \label{eq:linear_model_pix}
\end{equation}
where the pixels $\mathbf{p}'$ are close to $\mathbf{p}$ in a least-squares sense.
In what follows, we will assume that we work in the pixel basis and compute all fluxes in the $Y_{lm}$ basis where the model is exact.
We will discuss the validity of this approach in detail in Section~\ref{sec:inverse_problem}.

\subsection{Nonnegative Matrix Factorization}
\label{ssec:nmf}
In the case when we cannot assume a static map, the situation is considerably more complicated.
In principle, the surface map is different for each \emph{data point}.
Of course, fitting one map per data point is intractable so we might instead fit a single map per light curve, generated by Eq.~\ref{eq:linear_model_pix}.
Even then, we would need to fit on the order of 1k parameters!
This is both becuse the number of spherical coefficients scales as $(l+1)^2$ and because when fitting in pixel space we need many more pixels than spherical harmonics to ensure positivity.
Although fitting a model of such high dimensionality for a single IRTF light curve of high signal to noise ratio is tractable thanks to exact gradients provided by autodifferentiation in \textsf{starry}, we don't want to do that for a hundred or so separate light curves.
Such a model would be very difficult to fit and it would require strong regularization between successive maps in time.
It also wouldn't directly provide much physical insight into the volcanic activity, we would have to conduct a separate analysis on the inferred maps to constrain the time variability of the volcanoes.
Instead, we need an approach which reduces the dimensionality of the problem.
One way of accomplishing that is to expand the spherical harmonic coefficients (or pixels) in a Taylor series in time about a certain point.
This is the approach \cite{luger_tess_2019} took to model a map of Earth's albedo in reflected light from stray Earthshine in the aperture of the TESS space telescope.
This issue with this approach and similar series expansions such as the Fourier series is that with the very long baseline of IRTF data and the observations being scattered throughout the years sporadically, it is unclear which point should be the origin of the expansion.
More importantly, there likely doesn't exist a well defined global time-averaged map of Io so requiring too much smoothness between successive maps wouldn't work.
Another way of reducing the dimensionality of the problem would be a parametric approach where where we place a few spots on the sphere and then fit only for the locations and intensities of each spot.
This is assuming that we know a priori that the surface features are spot like and roughly how many we should expect.
While such a strong assumption might work for studying individual spots in the context of Io, it certainly isn't justified for exoplanets.
In Section~\ref{sec:inverse_problem} we do assume stronger priors on the locations of and shapes of surface features to better constrain the variability of individual volcanoes, but we still fit for pixels, at least in principle allowing the data to override our assumptions.

The approach approach opted for is to treat the problem of inferring a single map per light curve as a probabilistic matrix factorization problem.
The idea is to assume that a map for any given light curve can be expressed as a linear combination of $K$ "basis maps".
Consider $L$ lightcurves, the model for the $l$-th light curve is
\begin{equation}
    \mathbf{f}_l=\mathbf{A}_l\,\mathbf{P}^\dagger\,\mathbf{p}_l'
\end{equation}
We can stack the column vectors $\mathbf{p}_l$ into a matrix $\mathbf{Y}$ of shape $(N_p, L)$ where $N_p$ is the number of pixels for each light curve.
We then assume that the $\mathbf{Y}$ can be decomposed into a product of two matrices $\mathbf{B}$ and $\mathbf{Q}$ as
\begin{equation}
    \mathbb{Y}=\mathbf{B}\,\mathbf{Q}
    \label{eq:nmf}
\end{equation}
where $\mathbf{B}$ has shape $(N_p, K)$ and $\mathbf{Q}$ has shape $(K, L)$.
The model for all light curves can be written as
\begin{equation}
    \mathbb{f}=\mathbb{A}\,\mathrm{vec}(\mathbf{\mathbb{Y}})
    \label{eq:model_all_lcs}
\end{equation}
Where $\mathbf{f}$ is a tall column vector consisting of predictions for all light curves stacked together, $\mathbb{A}$ is a block diagonal matrix with matrices $\mathbf{A}_l\,\mathbf{P}^\dagger$ on the diagonal for $l=1\dots, L$ and the $\mathrm{vec}$ operator stacks the columns of $\mathbb{Y}$ into a tall vector.
The model is \emph{bilinear} which means that is linear in the matrices $\mathbf{B}$ and $\mathbf{Q}$ separately.
The interpretation of Equation~\ref{eq:nmf} is simple, each of the $K$ columns of $\mathbf{B}$ represents pixels of a basis map and the columns of $\mathbf{Q}$ determine how those maps add together to produce a final map for the $l$-th light curve.
For this reason we call $\mathbf{B}$ the \emph{basis matrix} and the matrix $\mathbf{Q}$ the \emph{encoding matrix}.
Ideally, we want the basis maps to be physically meaningful and the coefficients encode the time variability.
A nice feature of this model is that there is no requirement that successive maps are smoothly varying between different light curves.

The matrix factorization problem as written in Equation~\ref{eq:nmf} is highly degenerate and practically intractable a probabilistic framework with a small dataset.
Since each of the columns of $\mathbf{B}$ are pixels representing emitted light or albedo, we can substantially reduce the ambiguity in the decomposition by requiring that both matrices are strictly positive.
This matrix factorization problem is known as \emph{Nonnegative Matrix Factorization} (NMF) \citep{paatero_positive_1994,lee_algorithms_2001} and it has a rich history across many different fields.
It is commonly used for decomposing physical signals in the spectral or the time domain.
Even though NMF is more tractable than unconstrained matrix factorization, it is still an NP hard problem \citep{vavasis_complexity_2009} and it requires additional constraints or priors to be tractable.
Simultaneously transforming $\mathbf{B}\leftarrow \mathbf{B}\,\mathbf{S}^{-1}$ and
$\mathbf{Q}\leftarrow \,\mathbf{S}\mathbf{Q}$ with a nonsingular matrix $\mathbf{S}$ doesn't change the value of the objective function.
The objective function is also invariant to a simultaneous permutation of the columns of $\mathbf{B}$ and rows of $\mathbf{Q}$ and rescaling either of the two matrices by a scalar.
Nevertheless, getting around these degeneracies in practice is possible, especially when we have a lot of prior knowledge on the problem.
For a recent review article with a detalied analysis of these degeneracies and common algorithms, see \cite{fu_nonnegative_2019}.


Broadly, there are two approaches to fitting NMF models, the constrained optimization approach and the probabilistic approach.
In the optimization approach we optimize $\mathbb{Y}$ in a least-squares sense under a set of constraints on either one or both of the matrices \citep[see][for recent examples from the astronomical literature]{acosta-pulido_new_2017, melchior_scarlet_2018,ren_non-negative_2018,ren_using_2020} with the goal of obtaining a point estimate for the two matrices.
The probabilistic approach introduces priors and the inference is usually done with variational inference in order to obtain an approximate posterior distribution over the matrices.
In this work we opt for the probabilistic approach because we care about the uncertainties on the map features a lot.

One difference between most applications of NMF in the literature and NMF within the context of inferring surface maps from 1D light curves is that
in the majority of the work in the literature \citep[except][]{kawahara_global_2020} the matrix $\mathbb{Y}$ is assumed to be directly observed  up to a simple noise term.
In our case the problem is considerably more challenging because we observe the light curves
$\mathbb{A}\,\mathrm{vec}(\mathbf{\mathbb{Y}})$ instead of $\mathbb{Y}$ directly, a process in which some information is lost because, depending on the geometry of the occultations, certain combinations of spherical harmonic coefficients will be in the nullspace.
\cite{kawahara_global_2020} are using NMF to model phase variations in directly imaged exoplanets with the goal of simultaneously inferring a surface map and a spectral decomposition of the map into several components, a case in which the
matrix which is to be decomposed is also unobserved.
Although we don't attempt to solve the problem of spectrally decomposing the maps, our approach is directly applicable to that problem as well. (TODO: should I add more details here?)
One could also imagine inferring a multi spectral component and time dependent map by solving two NMF problems simultaneously.
We leave this application for future work.


\subsection{The likelihood}
\label{ssec:likelihood}
Given Eq.~\ref{eq:nmf} we can write down the likelihood for our data and in Section~\ref{sec:inverse_problem} we dicuss in detail the priors which make the problems of inferring static and dynamic maps tractable.
Assuming a Gaussian noise process for the data with a dense covariance matrix, the (log) likelihood is
\begin{equation}
    \ln\mathcal{L}=-\frac{1}{2}\left[\mathbb{f}_\mathrm{obs}-(\mathbb{f} + \mathbb{b})\right]^{\top} \boldsymbol{\Sigma}^{-1}\left[\mathbb{f}_\mathrm{obs}-(\mathbb{f} + \mathbb{b})\right]
\end{equation}
where $\mathbb{f}_\mathrm{obs}$ is the vector of stacked observed light curves, $\boldsymbol{\Sigma}$ is the data covariance matrix and $\mathbb{b}$ is a fixed flux offset per light curve accounting for stray flux not attributed to Io (mostly due to Jupiter).
We model the data covariance with a Gaussian Process using the fast Celerite method\citep{foreman-mackey_fast_2017} as implemented in the \textsf{exoplanet} package (TODO: cite) to capute correlations imposed by the seeing plus an additional white noise term which is different for each light curve.
The IRTF data isn't provided with any uncertainties so we estimate the uncertainties by filtering the data with the Savitzky Golay filter as implemented in \textsf{SciPy} (TODO: CITE) and estimating the variance of the residuals.
We assume identical errors for all data points in a given light curve but allow for a rescaling factor when fitting the model.
In the next section, we add priors to this likelihood and fit the model on simulated data using optimization and Variational Inference (VI).

%TODO: mention that A depends on the geometry and that we fit for some of those parameters.

\section{The inverse problem}
\label{sec:inverse_problem}
Having defined the forward models in the previous section, in this section we focus on choosing priors, doing inference and testing the models on simulated data.
We compute the information content of different kinds of occultation and phase light curves and , we test the differences between fitting in the $Y_{lm}$ basis vs. fitting in the pixel basis, prior choices for pixel maps and priors for NMF. 
To fit the models, we use  variational inference and Hamiltonian Monte Carlo.

% bulk of the results in the paper
\subsection{The information content of a light curve}
\label{ssec:information_content}
The mapping problem is famously ill posed.
In other words, certain linear combinations of spherical harmonic coefficients will be in the nullspace of the linear mapping  $\mathbf{A}$ in Eq.~\ref{eq:linear_model}.
This means that even if we had noiseless observations, certain features would still be impossible to recover.
To recover surface features from observed flux integrated over the disc of the object we need
to have some mechanism which breaks the various degeneracies.
For phase curves in emitted light, we can only recover longitudinal variations in intensity.
Occultations are a lot better thanks to the limb of the occultor sweeping across the surface of the occulted sphere.
Even better are grazing occultations with an occultor of a small diameter sweeping across the disc at different latitudes and therefore breaking the latitudinal degeneracies. 
Observing phase curves and occultations in reflected light improves the model even more because of the nonuniform illumination profile of the incident radiation and the presence of a day/night terminator line \citep[][Luger et al. 2020 in prep]{luger_starry_2019}.
In some cases for reflected light observations (phase curves of an inclined planet for example) there is no nullspace at all at low degrees of the map.

To compute how much information can in principle be obtained by observing occultations of Io by Jupiter and other moons or phase curves we need to compute some measure of how much does observing Io at certain times reduce our prior uncertainty on the map features.
Given that the model for a static map (Eq.~\ref{eq:linear_model}) is linear, under Gaussian priors with covariance $\boldsymbol{\Lambda}$ on the spherical harmonic coefficients and a Gaussian likelihood, the posterior is analytic and the mean is given by
\begin{equation}
    \widehat{\mathbf{y}}=\boldsymbol{\Sigma}_{\hat{\mathbf{y}}}\left(\mathbf{A}^{\top} \boldsymbol{\Sigma}_{\mathbf{f}}^{-1} \mathbf{f}+\boldsymbol{\Lambda}_{\mathbf{y}}^{-1} \boldsymbol{\mu}_{\mathbf{y}}\right)
\end{equation}
where the posterior covariance matrix $\boldsymbol{\Sigma}_{\hat{\mathbf{y}}}$ is 
\begin{equation}
\boldsymbol{\Sigma}_{\hat{\mathbf{y}}}=\left( \mathbf{A}^{\top} \boldsymbol{\Sigma}_{\mathbf{f}}^{-1} \mathbf{A} +\boldsymbol{\Lambda}_{\mathbf{y}}^{-1}\right)^{-1}
\end{equation}
The posterior covariance matrix depends on the design matrix $\mathbf{A}$, the prior covariance  $\boldsymbol{\Lambda}$ and the data covariance $\boldsymbol{\Sigma}_{\mathbf{f}}$.
We first compute the design matrices using \textsf{starry} for different kinds of events a over the course of a single year: occultations by Jupiter, occultations by Europa, Ganymede and Callisto and phase curves.

We take the ephemeris data  from \textsf{JPL HORIZONS} and assume that it is known exactly, we also assume all observations are of the emitted light component independent of whether Io is in sunlight or in eclipse.
This is because we are interested in constraining the volcanic emission rather than the albedo.
Given the fixed design matrices, we can compute the variance reduction of the posterior relative to the prior known as \emph{posterior shrinkage}, defined as
\begin{equation}
    1-\frac{\sigma^2_\mathrm{post}}{\sigma^2_\mathrm{prior}}
\end{equation}
where $\sigma^2_\mathrm{prior}$ is the prior and $\sigma^2_\mathrm{post}$ is the posterior variance for a given parameter.
To compute this quantity for each spherical harmonic coefficient in $\mathbf{y}$ we divide the diagonal elements of matrices $\boldsymbol{\Sigma}_{\hat{\mathbf{y}}}$ and $\boldsymbol{\Lambda}$, divide them, and average over all $m$-modes. 
We need to assume some variance for the Gaussian prior on the $Y_{l,m}$ coefficients so we choose a broad prior $\boldsymbol{\Lambda}=\mathrm{diag}(1^2,0.1^2,\dots,0.1^2)$.
Although the posterior shrinkage depends on this prior the main results do not change for a different choice of the prior.
The result is plotted in Fig.~\ref{fig:information_content}.
\begin{figure}[h!]
    \begin{centering}
    \includegraphics[width=0.5\linewidth]{figures/information_content.pdf}
    \oscaption{InformationContent}{
        The plot shows the information content of different kinds of observations of Io as a function of spherical harmonic degree (angular scale of surface features).
        On the y axis is posterior shrinkage (the variance ratio between the prior and the posterior distribution) for each spherical harmonic coefficient at a given degree $l$ on the x axis, averaged across all $m$ modes.
        Posterior shrinkage of 1 represent maximum information gain in updating from the prior to the posterior while a shrinkage of 0 represents no information gain. 
        We compute the posterior variance for different kinds of simulated observations of Io over
        the course of a single year: phase curves (blue), occultations by Jupiter (orange), combined phase curves and occultations by Jupiter (green) and occulltations of Io by other Galilean moons (red lines). 
        Solid lines represent observations at SNR=100 while dashed lines represent observations at SNR=10.
        \label{fig:information_content}
    }
    \end{centering}
\end{figure}
The plot shows posterior shrinkage for phase curve observations (blue lines), occultations by Jupiter (orange), the previous two combined (green) and mutual occultations by other Galilean moons (red). 
Solid lines correspond to SNR=100 while dashed lines correspond to SNR=10 where the signal is defined as the min-max difference of the predicted flux.

        It is clear that observations of mutual occultations of Io (red lines) are by far the most informative and most useful for constraining surface features. The posterior shrinkage is near unity at all angular scales $l$ up to $l=18$ owing to degeneracy breaking features of these occcultations. 
        Occultations by Jupiter (orange lines) are far less informative because we only see one side of Io during an occultation (hence the kink in the curve at $l=1$) and the position of Jupiter's limb on the surface of Io does not vary significantly so latitudinal features are only weakly constrained.
        The posterior shrinkage is exactly zero for phase curves at all odd degrees above $l=2$ because phase curves for objects rotating about an axis perpendicular to the line of sight encode zero information on these coefficients because they are in the nullspace.
Although observations of mutual occultations most easily break the degeneracies, they only occur every 6 years and more importantly they almost never occur while Io is in eclipse.
This means that only the brightest volcanoes will be visible above the reflected light background emission due to the Sun.
Figure~\ref{fig:information_content} does not reflect this fact.
When modeling the observations taken while Io is in sunlight, we have to take into account the reflected light component.
Although \textsf{starry} supports computing occultations and phase curves in reflected light, we only fit in eclipse observations in this work.

\subsection{Fitting a static map}
\label{ssec:static_map}
In this section we test inference with the static map model defined in Section~\ref{ssec:static} on both simulated and real data.
First we use \textsf{starry} to generate one simulated light curve of an occultation of Io by Jupiter with an $l=30$ map containing a single bright hot spot.
We take ephemeris parameters (assumed to be fixed) from a real occultation and assume an observational cadence and noise similar to IRTF light curves.
We use this dataset to test the difference between fitting a map in the spherical harmonic basis (Eq.~\ref{eq:linear_model}) and in the pixel basis (Eq.~\ref{eq:linear_model_pix}).
We fit two models for the Maximum A Posteriori (MAP) estimate of the parameters using the
\textsf{BFGS} algorithm as implemented in \textsf{scipy.optimize.minimize} in the \textsf{scipy} package (CITE) and the exact gradients of the model with respect to all parameters provided by \textsf{Theano}.

In the first model we place a Gaussian prior on $Y_{lm}$ coefficients with covariance $\boldsymbol{\Lambda}=\mathrm{diag}(1^2,0.1^2,\dots,0.1^2)$ and optimize for the values of the coefficients. 
In the second model we fit for pixels $\mathbf{p}'$ with a positive prior  in order to ensure (approximate) positivity of the inferred maps.
To evaluate the flux (Eq.~\ref{eq:linear_model_pix}) we map the pixels to a spherical harmonic coefficient vector $\mathbf{y}$ using the matrix $\mathbf{P}^\dagger$ at each step in the inference algorithm.
As the end product of the inference we do not store the interim pixels $\mathbf{p}'$ but rather the spherical harmonic coefficients. 
To distinguish between the pixels $\mathbf{p}'$ which are defined on a fixed grid in (latitude,longitude) and the pixels $\mathbf{p}=\mathbf{P}\,\mathbf{y}$ which can be computed on an arbitrary grid conditional on knowing $\mathbf{y}$, we call the former \emph{schmixels}.
Although the priors on schmixels $\mathbf{p}'$ are strictly positive, because of the approximate mapping to spherical harmonics the pixels $\mathbf{p}$ won't completely respect the prior and there will be still be some regions of the map with negative intensity.

The justification for this somewhat convoluted procedure is shown in Fig.~\ref{fig:pixels_vs_harmonics}.

\begin{figure}[h!]
    \begin{centering}
    \includegraphics[width=1.\linewidth]{figures/pixels_vs_harmonics.pdf}
    \oscaption{pixels_vs_harmonics}{%
        Maximum A Posteriori (MAP) estimates of $l=20$ maps (second panel from the top) produced by fitting a simulated light curve (third panel) which generated from an $l=30$ map consisting of a single bright spot (top panel). 
        The left portion of the figure shows a map inferred using a model in which we directly fit for spherical harmonic coefficients with a Gaussian prior on the coefficients.
        The right portion shows a map inferred from a model in which an Exponential prior is placed on the pixels defined on a fixed latitude-longitude grid which are converted to spherical harmonic coefficients at every step of the inference process.
        The pixels are only used as a tool to impose positivity and sparsity (via the Exponential prior) on the map intensity but the final inferred map (second panel on the right) and the predicted flux (orange lines) are generated from a map defined by a set of spherical harmonic coefficients. 
Both models fit the data reasonably well but the inferred maps are widely different. 
As a result of the sparse prior, the model whose prior is specified in pixel space results in a considerably simpler map with a single bright features where all the other regions of the map are pushed towards zero intensity.
Both maps show significant ringing effects due to the finite order of the spherical harmonic expansion. 
        The precise location of the true spot is not recovered for either of the two models
        which is not surprising because a single light curve is not sufficient to break the degeneracies.
        \label{fig:pixels_vs_harmonics}
    }
    \end{centering}
\end{figure}


\begin{figure}[h!]
    \begin{centering}
    \includegraphics[width=0.5\linewidth]{figures/pixels_hist.pdf}
    \oscaption{pixels_hist}{%
        Cumulative distribution of pixels in the true map and the inferred maps shown in Fig.~\ref{fig:pixels_vs_harmonics}.
        The pixels were obtained by dotting a pixel transform matrix $\mathbf{P}$ into the spherical coefficient vector $\mathbf{y}$ of each map.
        The top panel shows the distribution of pixels for the model in which we fit for the spherical harmonic coefficients with a Gaussian prior (orange). 
        The distribution of pixels in the true map is shown in blue.
        The bottom panel shows the pixels for a second model (right panel of Fig.~\ref{fig:pixels_vs_harmonics}) in which we fit for interim pixels which we call schmixels (green) instead of the spherical harmonic coefficients although the coefficients are still used to compute the flux and to compute the final pixel distribution (orange).
        The true map pixels are almost all positive (as they should be) except for a small tail due to truncation of the spherical harmonic expansion.
        The distribution of inferred pixels in the spherical harmonic model looks nothing like the truth and there is a significant left tail corresponding to negative values of map intensity which is due to the ringing artefacts which aren't sufficiently suppressed by the Gaussian prior.
        The distribution of schmixels (interim pixels used to enforce positivity of the map)       follows the exponential prior and is strictly positive by definition. 
        The distribution of corresponding pixels is not identical to that of schmixels because the mapping $\mathbf{P}$ from spherical harmonics to pixels is not invertible.
        However, the resulting distribution is closer the truth with a smaller left tail.
        \label{fig:pixels_hist}
    }
    \end{centering}
\end{figure}

To suppress ringing which results in regions of negative intensity surrounding a spot like feature as in Figure~\ref{fig:pixels_vs_harmonics} we choose to apply a spatial smoothing filter to the spherical harmonics.
Mathematically, the filtering operation is a convolution between the map and some kernel function $B(\theta,\phi)$.
The simplest choice for the kernel function is a Gaussian distribution given by
\begin{equation}
    B(\theta)=\frac{1}{\sqrt{2 \pi \sigma_s^{2}}}\exp \left(-\theta^{2} / 2 \sigma_s^{2}\right)
\end{equation}
where $\sigma_s$ sets the characteristic scale of the smoothing.
This function can be expended in terms of spherical harmonics as
\begin{equation}
B(\theta)=\sum_{l=0}^{\infty}\left(\frac{2 l+1}{4 \pi}\right) B_{l} P_{l}(\cos \theta)
\end{equation}
where $B_l$ are the spherical harmonic coefficients and $P_l$ are the associated Legendre polynomials.
They depend only on $l$ because the kernel function is azimuthally symmetric so all nonzero $m$ modes vanish.
For $\sigma_s\ll 1$ $B_l$ can be approximated as \citep{seon2007}
\begin{equation}
    B_l\approx \exp\left[-\frac{1}{2}l(l+1)\sigma_s^2\right]
\end{equation}
The Gaussian filter results in an exponential cutoff for scales smaller than $l\sim \sigma_s^{-1}$.


\begin{figure}[h!]
    \begin{centering}
    \includegraphics[width=1.\linewidth]{figures/smoothing_kernel.pdf}
    \oscaption{smoothing_kernel}{%
       Normalized intensity of a spherical harmonic expansion of 
        a Gaussian spot placed at zero degrees latitude and longitude.
        The expansion is in the quantity $\cos\Delta\theta$ where $\Delta\theta$ is the angular separation between the center of the spot and another point on the surface of the sphere, it is set to 5 degrees.
        The three insets show the spot profile at different strengths of the smoothing kernel
        where the leftmost spot has no smoothing ($\sigma_s=0$) and the rightmost spot is strongly smoothed. 
        The lines of different colors correspond to spot expansions up to certain order and the black line is the expansion at infinite order.
        The purpose of smoothing is to taper the higher order spherical harmonic coefficients.
        No smoothing results in significant "rings" where the intensity goes negative, even at high order.
        High levels of smoothing reduce the ringing but increase the spot size and eliminate differences between expansions above a certain order.
                \label{fig:smoothing_kernel}
    }
    \end{centering}
\end{figure}


\begin{figure}[h!]
    \begin{centering}
    \includegraphics[width=1.\linewidth]{figures/ingress_egress_simulated.pdf}
    \oscaption{ingress_egress_simulated}{%
        Maximum A Posteriori (MAP) estimate of an $l=25$ map (right panel from the top) produced by fitting a pair of light curves simulated from the true map which is shown on the left.
        The simulated light curves are shown in the third panel from the top, an ingress light curve on the left and an egress light curve on the right.
        The small circles above each of the light curves show the progression of the occultation over the inferred map.
        The precise location of the true spot is not recovered for either of the two models
        which is not surprising because a single light curve is not sufficient to break the degeneracies.
        \label{fig:ingress_egress_simulated}
    }
    \end{centering}
\end{figure}


\begin{figure}[h!]
    \begin{centering}
    \includegraphics[width=1.\linewidth]{figures/irtf_ingress_egress.pdf}
    \oscaption{irtf_ingress_egress}{%
        This is a plot of a pretty function. And at the end of this
        caption is a symbol with a link to the \emph{exact} script
        that generated it, hosted on \textsf{GitHub}.
        \label{fig:irtf_ingress_egress}
    }
    \end{centering}
\end{figure}


\begin{figure}[h!]
    \begin{centering}
    \includegraphics[width=.5\linewidth]{figures/irtf_ingress_egress_loki.pdf}
    \oscaption{irtf_ingress_egress_loki}{%
        This is a plot of a pretty function. And at the end of this
        caption is a symbol with a link to the \emph{exact} script
        that generated it, hosted on \textsf{GitHub}.
        \label{fig:irtf_ingress_egress_loki}
    }
    \end{centering}
\end{figure}


\subsection{Fitting a dynamic map}
\label{ssec:dynamic_map}
% Priors for NMF. Optimization vs. bayesian approach to NMF. Results on simulated data. Comparison of results to matrix factorization without the positivity constraint and clever priors.

\section{Results}
\label{sec:results}

\subsection{Individual events}
\label{ssec:individual}
% Fits of individual events. In particular, those light curves that were analyzed in previous papers.

\subsection{The time-variable map}
\label{ssec:time_variable_map}
%

\subsection{Variability of known hotspots}
\label{ssec:variability_hotspots}
% Plot inferred intensities of known hotspots as function of time, reference previous work.

\section{Mapping volcanic exoplanets}
\label{sec:exoplanets}
% Application to exoplanets. Fitting mock JWST observations.

\section{Conclusions}
\label{sec:conclusions}

% Bibliography
\bibliography{bib}
\end{document}
